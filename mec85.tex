\documentclass{article}

\title{MEC85 Fall 2020 Notes}
\author{Drake Lin}

\usepackage[margin=1in]{geometry}
\usepackage{setspace}
\usepackage{amsmath}
\usepackage[compact]{titlesec}
\onehalfspacing
\setcounter{secnumdepth}{0}
\setcounter{tocdepth}{1}
\titlespacing*{\section}{0pt}{8 ex}{1 ex}

\begin{document}
    \maketitle
    \tableofcontents
    \newpage

    \section{Chapter 2: Vectors}
    \subsection{Scalars}
    Scalars are positive/negative values specified by magnitude (length, time)
    
    \subsection{Vectors}
    Vectors have a magnitude + direction (force, position, moment)\\
    3D vectors: $||A|| = \sqrt{A_x^2+A_y^2+A_z^2}$\\
    Position vector: $\vec{AB} = (x_B-x_A)i + (y_B-y_A)i + (z_B-z_A)i$ \\
    Dot product: $A*B = ||\vec{A}||||\vec{B}||cos\theta = A_xB_x + A_yB_y + A_zB_z$



    \section{Chapter 3: Moment}

    \subsection{Moment}
    The tendency of body to rotate about point (torque)\\
    $\vec{M_o} = \vec{r} x \vec{F} =
        \begin{vmatrix}
            \vec{i} & \vec{j} & \vec{k} \\ r_x & r_y & r_z \\ F_x & F_y & F_z
        \end{vmatrix}$ 
    
    moment of a force $\vec{F}$ about position O

    $\vec{r}$ = position vector directed from O to any point on line of action of F\\
    Sum of many moments = $M_{Ro}=\sum \vec{r}x\vec{F}$\\
    Moment of force about an axis = $\vec{M_a}=\vec{u_a}*rxF=M_a*\vec{u_a}=
        \begin{vmatrix}
            \vec{u_{ax}} & \vec{u_{ay}} & \vec{u_{az}} \\ r_x & r_y & r_z \\ F_x & F_y & F_z
        \end{vmatrix}$ 
    
    \subsection{Cross Product}
    $\vec{C}=\vec{A}x\vec{B}=(ABsin\theta)\vec{u}_c= 
        \begin{vmatrix}
            \vec{i} & \vec{j} & \vec{k} \\ A_x & A_y & A_z \\ B_x & B_y & B_z
        \end{vmatrix}$

    $c=ABsin\theta$ (magnitude)
    
    $\vec{u}_c$ = direction = right hand rule 

    

    \section{Chapter 4: Rigid Body}
    Equilibrium of a rigid body: $\vec{F_x}=0, \vec{M_R}=0$

    \subsection{Equilibrium in 2D}
    Free Body Diagram: understand all the known/unknown forces + moments on the body \\
    Support Reaction: prevents translation of body by exerting force, prevents rotation by providing moment\\
    Common support reactions:
    \begin{itemize}
        \item roller:
        \item pin:
        \item fixed:
    \end{itemize}
    Other forces:
    \begin{itemize}
        \item Springs: $F=kx, x=\Delta l$ 
        \item Friction: $F_s=u_sN$
    \end{itemize}
    Two force members: Forces applied only at 2 points of member. For the member to be in equilibrium, the forces must be the same magnitude, 
    in the opposite direction of each other, but be in the same line of action.

    \subsection{Equilibrium in 3D}
    Common support reactions:
    \begin{itemize}
        \item roller:
        \item ball and socket:
        \item fixed:
    \end{itemize}
    Equations of Equilibrium:
    $\begin{cases}
        \sum F_x = 0 \\ \sum F_y = 0 \\ \sum F_z = 0
    \end{cases}$
    $\begin{cases}
        \sum M_x = 0 \\ \sum M_y = 0 \\ \sum M_z = 0
    \end{cases}$



    \section{Chapter 5: Structural Analysis}
    Designing structures. Goal = maximize bending stiffness: change material or structure\\
    Bending stiffness = $R = \rho / \delta$ 
    
    \subsection{Trusses}
    Trusses are structures designed to suport loads such as roofs or buildings. They are slender members and joined at endpoints with each other.
    Its weight is minimized to maximize its strength to weight ratio. When analyzing trusses, we make 4 key assumptions:
    \begin{itemize}
        \item all loadings applied at joints
        \item weight of bar ignored
        \item joined together by smooth points
        \item consists of two force members
    \end{itemize}
    For truss analysis:
    \begin{enumerate}
        \item determine support reactions
        \item determine zero force members (members that carry zero force)
        \item determine forces supported by individual members of truss
    \end{enumerate}
    
    \subsection{Method of Joints}
    With the method of joints, we evaluate individual joints or pin connections. We assume tension in members and solve by summing forces in the x-y direction.

    \subsection{Method of Sections}
    With the method of sections, we evaluate a section including multiple joints by cutting through structural members. This section must be in equilibrium so $\sum F=0$ and  $\sum M=0$. 
    We cut a section to reduce the number of unknown variables.

    

    \section{Chapter 6: Center and Moment}
    

\end{document}
